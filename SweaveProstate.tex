\documentclass{article}\usepackage[]{graphicx}\usepackage[]{color}
%% maxwidth is the original width if it is less than linewidth
%% otherwise use linewidth (to make sure the graphics do not exceed the margin)
\makeatletter
\def\maxwidth{ %
  \ifdim\Gin@nat@width>\linewidth
    \linewidth
  \else
    \Gin@nat@width
  \fi
}
\makeatother

\definecolor{fgcolor}{rgb}{0.345, 0.345, 0.345}
\newcommand{\hlnum}[1]{\textcolor[rgb]{0.686,0.059,0.569}{#1}}%
\newcommand{\hlstr}[1]{\textcolor[rgb]{0.192,0.494,0.8}{#1}}%
\newcommand{\hlcom}[1]{\textcolor[rgb]{0.678,0.584,0.686}{\textit{#1}}}%
\newcommand{\hlopt}[1]{\textcolor[rgb]{0,0,0}{#1}}%
\newcommand{\hlstd}[1]{\textcolor[rgb]{0.345,0.345,0.345}{#1}}%
\newcommand{\hlkwa}[1]{\textcolor[rgb]{0.161,0.373,0.58}{\textbf{#1}}}%
\newcommand{\hlkwb}[1]{\textcolor[rgb]{0.69,0.353,0.396}{#1}}%
\newcommand{\hlkwc}[1]{\textcolor[rgb]{0.333,0.667,0.333}{#1}}%
\newcommand{\hlkwd}[1]{\textcolor[rgb]{0.737,0.353,0.396}{\textbf{#1}}}%
\let\hlipl\hlkwb

\usepackage{framed}
\makeatletter
\newenvironment{kframe}{%
 \def\at@end@of@kframe{}%
 \ifinner\ifhmode%
  \def\at@end@of@kframe{\end{minipage}}%
  \begin{minipage}{\columnwidth}%
 \fi\fi%
 \def\FrameCommand##1{\hskip\@totalleftmargin \hskip-\fboxsep
 \colorbox{shadecolor}{##1}\hskip-\fboxsep
     % There is no \\@totalrightmargin, so:
     \hskip-\linewidth \hskip-\@totalleftmargin \hskip\columnwidth}%
 \MakeFramed {\advance\hsize-\width
   \@totalleftmargin\z@ \linewidth\hsize
   \@setminipage}}%
 {\par\unskip\endMakeFramed%
 \at@end@of@kframe}
\makeatother

\definecolor{shadecolor}{rgb}{.97, .97, .97}
\definecolor{messagecolor}{rgb}{0, 0, 0}
\definecolor{warningcolor}{rgb}{1, 0, 1}
\definecolor{errorcolor}{rgb}{1, 0, 0}
\newenvironment{knitrout}{}{} % an empty environment to be redefined in TeX

\usepackage{alltt}

\usepackage{rotating}
\usepackage{graphics}
\usepackage{latexsym}
\usepackage{color}
\usepackage{listings} % allows for importing code scripts into the tex file
\usepackage{wrapfig} % allows wrapping text around a figure
\usepackage{lipsum} % provides Latin text to fill up a page in this illustration (do not need it otherwise!)

% Approximately 1 inch borders all around
\setlength\topmargin{-.56in}
\setlength\evensidemargin{0in}
\setlength\oddsidemargin{0in}
\setlength\textwidth{6.49in}
\setlength\textheight{8.6in}

% Options for code listing; from Patrick DeJesus, October 2016
\definecolor{codegreen}{rgb}{0,0.6,0}
\definecolor{codegray}{rgb}{0.5,0.5,0.5}
\definecolor{codepurple}{rgb}{0.58,0,0.82}
\definecolor{backcolour}{rgb}{0.95,0.95,0.92}
\lstdefinestyle{mystyle}{
	backgroundcolor=\color{backcolour},   commentstyle=\color{codegreen},
	keywordstyle=\color{magenta},
	numberstyle=\tiny\color{codegray},
	stringstyle=\color{codepurple},
	basicstyle=\footnotesize,
	breakatwhitespace=false,         
	breaklines=true,                 
	captionpos=b,                    
	keepspaces=true,                 
	numbers=left,                    
	numbersep=5pt,                  
	showspaces=false,                
	showstringspaces=false,
	showtabs=false,                  
	tabsize=2
}

%"mystyle" code listing set
\lstset{style=mystyle}
%\lstset{inputpath=appendix/}


\title{Detection of Prostate Cancer} 
\author{Kelso Quan}
\IfFileExists{upquote.sty}{\usepackage{upquote}}{}
\begin{document} 

\maketitle
\begin{center}
\Large{Abstract}
\end{center}

Prostate cancer is one of the most common cancers among men. Luckily, prostate cancer is treatable when detected early enough. The Ohio State University Comprehensive Cancer Center wants to know if baseline exam measurements can predict whether a tumor has penetrated the prostatic capsule indicting there is indeed a tumor present. The Prostatic Specific Antigen value (PSA), results of digital exam (dpros), and Gleason score has the ability of predicting whether a patient has a tumor penetrating the prostatic capsule. In the end, this model does not have any interaction terms even though they were considered within the scope of the analysis. 

\section{Introduction}
\qquad Since prostate cancer is one of the most common cancer among men, a study was conducted at the Ohio State University Comprehensive Cancer Center tried to determine if a tumor has penetrated the prostatic capsule. This is good news for a majority of men. Prostate cancer which occurs in about 1 out of 7 men, but is only 1 in 39 men will die due to this cancer according to webmd. Several factors contribute to cancer tumors penetrating the prostatic capsule which include: age of subject, race, results of digital exam, detection of capsular involvement, Prostatic Specific Antigen value, and total Gleason score. It appears that the Prostatic Specific Antigen value (PSA), results of digital exam, and Gleason score is able to predict whether a patient has a cancerous tumor penetrating the prostatic capsule. 



\section{Methods}
\qquad There was a cancer study on 380 male patients of either white or black race. Patients 1162, 1186, 1392 were excluded because those patients had at least one ``na" value listed. For those who do not know about prostate cancer, many of the variables will seem unfamiliar. PSA is a measure of protein produced by prostate gland cells and is measured in $mg/mL$. Elevated levels of PSA may suggest prostate cancer and is used as a screening test. The Gleason score is a scale from 1 to 10 measuring the abnormality of cells. Larger values of Gleason score suggest a higher risk of cancer. Race has two factors, whether the patient is black or white. The variable capsule indicates whether the tumor penetrated the prostatic capsule. dpros are the results of the digital rectal exam which can have no nodule, unilobar nodule, bilobar nodule. dcaps is the detection of capsular involvement. There was 153 of the 380 subjects who had a cancer that penetrated the capsule. 

\qquad This analysis will not have any transformation, but will consider interaction terms. 
The analysis will be conducted using R/RStudio. 


\section{Results}
\subsection{Exploratory Data Analysis}

\qquadIn In figure \ref{boxplot}, it shows two things, the amount of PSA when the capsule is penetrated and the distribution of having the capsule penetrated given the level of abnormality in the cells. This did show some interesting results. PSA is higher when the capsule is penetrated and the Gleason score is generally higher when there is penetration. These two graphs shows that the doctors and scientists at Ohio State were onto something. Also, a simple Chi-squared t-test located in the appendix would show that dpros has a significant association with the capsule being punctured.


\end{document}
